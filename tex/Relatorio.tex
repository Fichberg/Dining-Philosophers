\documentclass[11pt]{article}
\usepackage[brazil]{babel}
\usepackage[utf8]{inputenc}
\usepackage[usenames,dvipsnames,svgnames,table]{xcolor}
\usepackage[a4paper,margin={1in}]{geometry}
\usepackage{graphicx}
\usepackage{color}
\usepackage{pifont}
\usepackage{textcomp}
\usepackage{caption}
\usepackage{pgfplots}

\definecolor{sblue}{rgb}{0, 0, 1}
\definecolor{blue}{rgb}{0, 0.55, 1}
\definecolor{red}{rgb}{1, 0, 0}
\newcommand{\quotes}[1]{``#1''}

\begin{document}

\section{Sobre este documento}
\indent\indent Neste relatório, o leitor encontrará informações e resultados acerca dos testes realizados com o programa desenvolvido para resolver o problema dos \textit{filósofos famintos}.

\section{Configuração da máquina}
Processor: Intel\textregistered Core\texttrademark i5-3450 CPU @ 3.10GHz $\times$ 4 \\
HD Seagate 7200RPM 1TB \\
OS: Ubuntu 12.04 LTS\\
OS type: 32-bit

\section{Informações}

O programa foi testado para os seguintes valores de entrada:
\begin{itemize}
	\item \textbf{Entrada 1: Arquivo de entrada \textcolor{blue}{simple.txt}}:
	\begin{itemize}
		\item \textbf{Quantidade de filósofos}: 5
		\item \textbf{Pesos}: 10 1 6 5 1
		\item \textbf{Quantidade de comida \textcolor{blue}{$R_s$}}: 1000
	\end{itemize}
	\item \textbf{Entrada 2: Arquivo de entrada \textcolor{blue}{complex.txt}}:
	\begin{itemize}
		\item \textbf{Quantidade de filósofos}: 30
		\item \textbf{Pesos}: 5 3 4 3 4 6 4 7 2 4 3 3 7 1 4 2 3 2 10 4 5 6 3 5 5 7 9 2 5 4
		\item \textbf{Quantidade de comida \textcolor{blue}{$R_c$}}: 10000
	\end{itemize}
\end{itemize}

\noindent \textbf{Obs}: pesos ordenados em ordem crescente com relação aos identificadores dos filósofos. \\
\noindent \textbf{Para cada entrada}, o programa foi rodado \textbf{8 vezes}. No final, sera apresentado um gráfico com a média de porções consumidas por filósofo em todas as execuções.\\

\section{Tabelas de resultados das execuções}

As tabelas 1 e 2 são referentes às 8 execuções do arquivo \textbf{\textcolor{blue}{simple.txt}}. Já as tabelas 3 e 4, são referentes às 8 execuções do arquivo \textbf{\textcolor{blue}{complex.txt}}. Nas tabelas a seguir, \textbf{$C_i$} representa a quantidade de comida consumida na \textit{i-ésima} execução.

\begin{table}[!h]
	\begin{center}
		\begin{tabular}{| c | c | c | c | c | c | c | c | c |}
		\hline
		\multicolumn{9}{|c|}{\textbf{Tabela 1}} \\ \hline
		\multicolumn{9}{|c|}{\textbf{Arquivo \textcolor{blue}{simple.txt}} \& \textbf{\textcolor{blue}{$R_s$}} = 1000 \& \textbf{modo} = \textbf{\textcolor{blue}{U}}} \\
		\hline
			\textbf{\# Filósofo} & \textbf{$C_1$} & \textbf{$C_2$} & \textbf{$C_3$} & \textbf{$C_4$} & \textbf{$C_5$} & \textbf{$C_6$} & \textbf{$C_7$} & \textbf{$C_8$} \\ \hline
			1 & 190 & 192 & 203 & 201 & 196 & 202 & 202 & 204 \\ \hline
			2 & 199 & 200 & 200 & 203 & 202 & 199 & 197 & 204 \\ \hline
			3 & 203 & 210 & 193 & 200 & 198 & 201 & 198 & 202 \\ \hline
			4 & 204 & 204 & 190 & 204 & 205 & 196 & 199 & 203 \\ \hline
			5 & 204 & 194 & 214 & 192 & 199 & 202 & 204 & 187 \\ \hline
		\end{tabular}
	\end{center}
\end{table}

\pagebreak

\begin{table}[!h]
	\begin{center}
		\begin{tabular}{| c | c | c | c | c | c | c | c | c |}
		\hline
		\multicolumn{9}{|c|}{\textbf{Tabela 2}} \\ \hline
		\multicolumn{9}{|c|}{\textbf{Arquivo \textcolor{blue}{simple.txt}} \& \textbf{\textcolor{blue}{$R_s$}} = 1000 \& \textbf{modo} = \textbf{\textcolor{blue}{P}}} \\
		\hline
			\textbf{\# Filósofo} & \textbf{$C_1$} & \textbf{$C_2$} & \textbf{$C_3$} & \textbf{$C_4$} & \textbf{$C_5$} & \textbf{$C_6$} & \textbf{$C_7$} & \textbf{$C_8$} \\ \hline
			1 & 400 & 430 & 470 & 436 & 423 & 460 & 455 & 475 \\ \hline
			2 & 49  & 44  & 44  & 51  & 47  & 41  & 39  & 38  \\ \hline
			3 & 264 & 258 & 250 & 252 & 276 & 258 & 234 & 252 \\ \hline
			4 & 239 & 225 & 195 & 215 & 215 & 199 & 225 & 195 \\ \hline
			5 & 48  & 43  & 41  & 46  & 39  & 42  & 47  & 40  \\ \hline
		\end{tabular}
	\end{center}
\end{table}

\begin{table}[!h]
	\begin{center}
		\begin{tabular}{| c | c | c | c | c | c | c | c | c |}
		\hline
		\multicolumn{9}{|c|}{\textbf{Tabela 3}} \\ \hline
		\multicolumn{9}{|c|}{\textbf{Arquivo \textcolor{blue}{complex.txt}} \& \textbf{\textcolor{blue}{$R_c$}} = 10000 \& \textbf{modo} = \textbf{\textcolor{blue}{U}}} \\
		\hline
			\textbf{\# Filósofo} & \textbf{$C_1$} & \textbf{$C_2$} & \textbf{$C_3$} & \textbf{$C_4$} & \textbf{$C_5$} & \textbf{$C_6$} & \textbf{$C_7$} & \textbf{$C_8$} \\ \hline
			1  & 326 & 337 & 324 & 332 & 331 & 346 & 346 & 339 \\ \hline
			2  & 339 & 338 & 326 & 336 & 327 & 338 & 327 & 343 \\ \hline
			3  & 341 & 329 & 339 & 341 & 345 & 329 & 329 & 335 \\ \hline
			4  & 341 & 336 & 344 & 335 & 332 & 320 & 337 & 332 \\ \hline
			5  & 341 & 340 & 329 & 344 & 330 & 335 & 328 & 333 \\ \hline
			6  & 326 & 332 & 332 & 344 & 335 & 339 & 332 & 335 \\ \hline
			7  & 317 & 339 & 326 & 337 & 336 & 323 & 331 & 336 \\ \hline
			8  & 329 & 339 & 334 & 344 & 318 & 324 & 362 & 324 \\ \hline
			9  & 327 & 332 & 336 & 337 & 324 & 350 & 344 & 330 \\ \hline
			10 & 335 & 326 & 347 & 323 & 342 & 322 & 324 & 343 \\ \hline
			11 & 327 & 332 & 339 & 337 & 339 & 347 & 330 & 313 \\ \hline
			12 & 352 & 354 & 334 & 343 & 325 & 346 & 316 & 332 \\ \hline
			13 & 329 & 317 & 340 & 324 & 341 & 321 & 338 & 335 \\ \hline
			14 & 348 & 337 & 315 & 313 & 343 & 336 & 311 & 327 \\ \hline
			15 & 329 & 337 & 314 & 319 & 319 & 336 & 339 & 333 \\ \hline
			16 & 338 & 333 & 323 & 322 & 339 & 328 & 317 & 320 \\ \hline
			17 & 341 & 325 & 340 & 338 & 322 & 322 & 348 & 358 \\ \hline
			18 & 338 & 339 & 325 & 324 & 312 & 347 & 331 & 338 \\ \hline
			19 & 338 & 327 & 348 & 338 & 336 & 335 & 344 & 347 \\ \hline
			20 & 328 & 316 & 307 & 327 & 352 & 328 & 331 & 326 \\ \hline
			21 & 323 & 338 & 348 & 334 & 336 & 341 & 344 & 327 \\ \hline
			22 & 333 & 331 & 349 & 322 & 333 & 337 & 343 & 337 \\ \hline
			23 & 321 & 325 & 355 & 325 & 339 & 339 & 334 & 334 \\ \hline
			24 & 323 & 338 & 334 & 355 & 340 & 322 & 333 & 342 \\ \hline
			25 & 340 & 339 & 326 & 353 & 317 & 349 & 331 & 340 \\ \hline
			26 & 322 & 326 & 331 & 339 & 337 & 331 & 338 & 326 \\ \hline
			27 & 329 & 339 & 328 & 328 & 332 & 317 & 343 & 329 \\ \hline
			28 & 336 & 331 & 331 & 325 & 341 & 322 & 322 & 332 \\ \hline
			29 & 347 & 324 & 337 & 330 & 343 & 334 & 320 & 326 \\ \hline
			30 & 336 & 344 & 339 & 331 & 334 & 336 & 327 & 328 \\ \hline
		\end{tabular}
	\end{center}
\end{table}

\pagebreak

\begin{table}[!h]
	\begin{center}
		\begin{tabular}{| c | c | c | c | c | c | c | c | c |}
		\hline
		\multicolumn{9}{|c|}{\textbf{Tabela 4}} \\ \hline
		\multicolumn{9}{|c|}{\textbf{Arquivo \textcolor{blue}{complex.txt}} \& \textbf{\textcolor{blue}{$R_c$}} = 10000 \& \textbf{modo} = \textbf{\textcolor{blue}{P}}} \\
		\hline
			\textbf{\# Filósofo} & \textbf{$C_1$} & \textbf{$C_2$} & \textbf{$C_3$} & \textbf{$C_4$} & \textbf{$C_5$} & \textbf{$C_6$} & \textbf{$C_7$} & \textbf{$C_8$} \\ \hline
			1  & 360 & 380 & 375 & 395 & 380 & 375 & 350 & 370 \\ \hline
			2  & 261 & 228 & 213 & 225 & 234 & 213 & 246 & 222 \\ \hline
			3  & 296 & 292 & 336 & 312 & 300 & 292 & 328 & 268 \\ \hline
			4  & 246 & 216 & 222 & 216 & 222 & 219 & 231 & 243 \\ \hline
			5  & 304 & 316 & 304 & 328 & 268 & 320 & 320 & 288 \\ \hline
			6  & 456 & 432 & 474 & 468 & 468 & 414 & 390 & 498 \\ \hline
			7  & 300 & 324 & 336 & 328 & 280 & 296 & 316 & 312 \\ \hline
			8  & 574 & 574 & 532 & 532 & 525 & 511 & 567 & 525 \\ \hline
			9  & 152 & 142 & 136 & 146 & 150 & 144 & 144 & 140 \\ \hline
			10 & 288 & 296 & 336 & 312 & 328 & 312 & 332 & 304 \\ \hline
			11 & 252 & 203 & 228 & 225 & 237 & 234 & 216 & 243 \\ \hline
			12 & 231 & 228 & 237 & 216 & 243 & 240 & 228 & 237 \\ \hline
			13 & 546 & 525 & 581 & 518 & 504 & 511 & 518 & 525 \\ \hline
			14 & 79  & 72  & 72  & 74  & 76  & 79  & 75  & 74  \\ \hline
			15 & 296 & 312 & 296 & 316 & 316 & 304 & 300 & 296 \\ \hline
			16 & 150 & 166 & 162 & 160 & 154 & 146 & 146 & 148 \\ \hline
			17 & 210 & 240 & 222 & 216 & 216 & 219 & 225 & 261 \\ \hline
			18 & 160 & 158 & 148 & 144 & 164 & 154 & 156 & 160 \\ \hline
			19 & 750 & 800 & 680 & 700 & 820 & 750 & 800 & 810 \\ \hline
			20 & 292 & 304 & 324 & 312 & 292 & 349 & 324 & 292 \\ \hline
			21 & 385 & 350 & 385 & 385 & 390 & 415 & 365 & 375 \\ \hline
			22 & 402 & 450 & 426 & 420 & 490 & 438 & 456 & 474 \\ \hline
			23 & 231 & 198 & 243 & 252 & 231 & 216 & 237 & 216 \\ \hline
			24 & 430 & 390 & 401 & 390 & 380 & 365 & 340 & 370 \\ \hline
			25 & 325 & 415 & 440 & 355 & 360 & 370 & 375 & 360 \\ \hline
			26 & 508 & 497 & 469 & 525 & 546 & 609 & 518 & 539 \\ \hline
			27 & 729 & 711 & 603 & 693 & 639 & 648 & 648 & 648 \\ \hline
			28 & 142 & 148 & 164 & 148 & 144 & 648 & 150 & 150 \\ \hline
			29 & 325 & 365 & 355 & 365 & 355 & 395 & 390 & 360 \\ \hline
			30 & 320 & 268 & 300 & 324 & 288 & 292 & 309 & 292 \\ \hline
		\end{tabular}
	\end{center}
\end{table}


As médias de consumo $\mu$ de cada filósofo $F_i, 1 \leq i \leq 30$ das $T_j$ tabelas anteriores, $1 \leq j \leq 4$ são apresentadas em duas novas tabelas a seguir, uma para cada arquivo de entrada:

\begin{table}[!h]
	\begin{center}
		\begin{tabular}{| c | c | c |}
		\hline
		\multicolumn{3}{|c|}{\textbf{Tabela 5}} \\ \hline
		\multicolumn{3}{|c|}{\textbf{Arquivo \textcolor{blue}{simple.txt}} \& \textbf{\textcolor{blue}{$R_s$}} = 1000  } \\ \hline
			$\mu$ & $T_1$ (\textbf{modo} = \textbf{\textcolor{blue}{U}}) & $T_2$ (\textbf{modo} = \textbf{\textcolor{blue}{P}}) \\ \hline 
			$F_1$ & 198.750 & 443.625 \\ \hline
			$F_2$ & 200.500 & 44.125  \\ \hline
			$F_3$ & 200.625 & 255.500 \\ \hline
			$F_4$ & 200.625 & 213.500 \\ \hline
			$F_5$ & 199.500 & 43.250  \\ \hline	
		\end{tabular}
	\end{center}
\end{table}

\pagebreak

\begin{table}[!h]
	\begin{center}
		\begin{tabular}{| c | c | c |}
		\hline
		\multicolumn{3}{|c|}{\textbf{Tabela 6}} \\ \hline
		\multicolumn{3}{|c|}{\textbf{Arquivo \textcolor{blue}{complex.txt}} \& \textbf{\textcolor{blue}{$R_c$}} = 10000  } \\ \hline
			$\mu$ & $T_3$ (\textbf{modo} = \textbf{\textcolor{blue}{U}}) & $T_4$ (\textbf{modo} = \textbf{\textcolor{blue}{P}}) \\ \hline 
			$F_1$ & 335.125 & 373.125 \\ \hline
			$F_2$ & 334.250 & 230.250 \\ \hline
			$F_3$ & 336.000 & 303.000 \\ \hline
			$F_4$ & 334.625 & 226.875 \\ \hline
			$F_5$ & 335.000 & 306.000 \\ \hline
			$F_6$ & 334.375 & 450.000 \\ \hline
			$F_7$ & 330.625 & 311.500 \\ \hline
			$F_8$ & 334.250 & 542.500 \\ \hline
			$F_9$ & 335.000 & 144.250 \\ \hline
			$F_{10}$ & 332.750 & 313.500 \\ \hline
			$F_{11}$ & 333.000 & 229.750 \\ \hline
			$F_{12}$ & 337.750 & 232.500 \\ \hline
			$F_{13}$ & 330.625 & 528.500 \\ \hline
			$F_{14}$ & 328.750 & 75.125  \\ \hline
			$F_{15}$ & 328.250 & 304.500 \\ \hline
			$F_{16}$ & 327.500 & 154.000 \\ \hline
			$F_{17}$ & 336.750 & 226.125 \\ \hline
			$F_{18}$ & 331.750 & 155.500 \\ \hline
			$F_{19}$ & 339.125 & 763.750 \\ \hline
			$F_{20}$ & 326.875 & 311.125 \\ \hline
			$F_{21}$ & 336.375 & 381.250 \\ \hline
			$F_{22}$ & 335.625 & 444.500 \\ \hline
			$F_{23}$ & 334.000 & 228.000 \\ \hline
			$F_{24}$ & 335.875 & 383.250 \\ \hline
			$F_{25}$ & 336.875 & 375.000 \\ \hline
			$F_{26}$ & 331.250 & 526.375 \\ \hline
			$F_{27}$ & 330.625 & 664.875 \\ \hline
			$F_{28}$ & 330.000 & 211.750 \\ \hline
			$F_{29}$ & 332.625 & 363.750 \\ \hline
			$F_{30}$ & 334.375 & 299.125 \\ \hline
		\end{tabular}
	\end{center}
\end{table}

\section{Gráficos}

Uma vez conhecidas as médias (Tabelas 5 e 6), podemos obter os seguintes gráficos de barras que ilustram a distribuição das porções de comida \textbf{\textcolor{blue}{$R_c$}} e \textbf{\textcolor{blue}{$R_s$}}. Cada valor da abscissa representa um filósofo.

\pagebreak

Nos gráficos a seguir, estão impressos apenas os 2 primeiros digitos decimais de cada valor médio $\mu$ encontrado.

\begin{center}
	\begin{figure}[!h]
		{\caption*{Gráfico formado com valores médios $\mu$ obtidos anteriormente para o arquivo \textbf{\textcolor{blue}{simple.txt}} no modo \textbf{\textcolor{blue}{U}} com \textbf{\textcolor{blue}{$R_s$}} = 1000}}
		\begin{tikzpicture} 
			\begin{axis}[ ymin=0, ymax=250 ,ybar, enlargelimits=0.15, legend style={at={(0.5,-0.20)}, anchor=north,legend columns=-1}, 
				ylabel={Consumo médio $\mu$}, xlabel={Filósofos}, symbolic x coords={1,2,3,4,5}, xtick=data, nodes near coords, 
				nodes near coords align={vertical}, x post scale=2.2, every node near coord/.append style={font=\tiny, rotate=90, anchor=west},bar width=15pt
			] 
			\addplot coordinates {(1,198.750) (2,200.500) (3,200.625) (4,200.625) (5,199.500)};
			\legend{Comida consumida pelo filósofo}
			\end{axis}
		\end{tikzpicture}
	\end{figure}
\end{center}

\begin{center}
	\begin{figure}[!h]
		{\caption*{Gráfico formado com valores médios $\mu$ obtidos anteriormente para o arquivo \textbf{\textcolor{blue}{simple.txt}} no modo \textbf{\textcolor{blue}{U}} com \textbf{\textcolor{blue}{$R_s$}} = 1000}}
		\begin{tikzpicture} 
			\begin{axis}[ ymin=0, ymax=500 ,ybar, enlargelimits=0.15, legend style={at={(0.5,-0.20)}, anchor=north,legend columns=-1}, 
				ylabel={Consumo médio $\mu$}, xlabel={Filósofos}, symbolic x coords={1,2,3,4,5}, xtick=data, nodes near coords, 
				nodes near coords align={vertical}, x post scale=2.2, every node near coord/.append style={font=\tiny, rotate=90, anchor=west},bar width=15pt
			] 
			\addplot coordinates {(1,443.625) (2,44.125) (3,255.500) (4,213.500) (5,43.250)};
			\legend{Comida consumida pelo filósofo}
			\end{axis}
		\end{tikzpicture}
	\end{figure}
\end{center}

\pagebreak


\begin{center}
	\begin{figure}[!h]
		{\caption*{Gráfico formado com valores médios $\mu$ obtidos anteriormente para o arquivo \textbf{\textcolor{blue}{complex.txt}} no modo \textbf{\textcolor{blue}{U}} com \textbf{\textcolor{blue}{$R_c$}} = 10000}}
		\begin{tikzpicture} 
			\begin{axis}[ ymin=0, ymax=400 ,ybar, enlargelimits=0.08, legend style={at={(0.5,-0.20)}, anchor=north,legend columns=-1}, 
				ylabel={Consumo médio $\mu$}, xlabel={Filósofos}, symbolic x coords={1,2,3,4,5,6,7,8,9,10,11,12,13,14,15,16,17,18,19,20,21,22,23,24,25,26,27,28,29,30}, xtick=data, nodes near coords, 
				nodes near coords align={vertical}, x post scale=2.2, every node near coord/.append style={font=\tiny, rotate=90, anchor=west},bar width=6pt
			] 
			\addplot coordinates {(1,335.125) (2,334.250) (3,336.000) (4,334.625) (5,335.000) (6,334.375) (7,330.625) (8,334.250) (9,335.000) (10,332.750) (11,333.000) (12,337.750) (13,330.625) (14,328.750) (15,328.250) (16,327.500) (17,336.750) (18,331.750) (19,339.125) (20,326.875) (21,336.375) (22,335.625) (23,334.000) (24,335.875) (25,336.875) (26,331.250) (27,330.625) (28,330.000) (29,332.625) (30,334.375)};
			\legend{Comida consumida pelo filósofo}
			\end{axis}
		\end{tikzpicture}
	\end{figure}
\end{center}

\begin{center}
	\begin{figure}[!h]
		{\caption*{Gráfico formado com valores médios $\mu$ obtidos anteriormente para o arquivo \textbf{\textcolor{blue}{complex.txt}} no modo \textbf{\textcolor{blue}{P}} com \textbf{\textcolor{blue}{$R_c$}} = 10000}}
		\begin{tikzpicture} 
			\begin{axis}[ ymin=0, ymax=860 ,ybar, enlargelimits=0.08, legend style={at={(0.5,-0.20)}, anchor=north,legend columns=-1}, 
				ylabel={Consumo médio $\mu$}, xlabel={Filósofos}, symbolic x coords={1,2,3,4,5,6,7,8,9,10,11,12,13,14,15,16,17,18,19,20,21,22,23,24,25,26,27,28,29,30}, xtick=data, nodes near coords, 
				nodes near coords align={vertical}, x post scale=2.2, every node near coord/.append style={font=\tiny, rotate=90, anchor=west},bar width=6pt
			] 
			\addplot coordinates {(1,373.125) (2,230.250) (3,303.000) (4,226.875) (5,306.000) (6,450.000) (7,311.500) (8,542.500) (9,144.250) (10,313.500) (11,229.750) (12,232.500) (13,528.500) (14,75.125) (15,304.500) (16,154.000) (17,226.125) (18,155.500) (19,763.750) (20,311.125) (21,381.250) (22,444.500) (23,228.000) (24,383.250) (25,375.000) (26,526.375) (27,664.875) (28,211.750) (29,363.750) (30,299.125)};
			\legend{Comida consumida pelo filósofo}
			\end{axis}
		\end{tikzpicture}
	\end{figure}
\end{center}

\section{Comentários}

Os resultados foram os esperados. Assim como mencionado no enunciado do exercício, os resultados para os modos \textbf{\textcolor{blue}{U}} e \textbf{\textcolor{blue}{P}} deram, respectivamente, barras de praticamente o mesmo tamanho no primeiro caso e barras proporcionais aos pesos dos filósofos no segundo. 

\end{document}