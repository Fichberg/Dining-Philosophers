\documentclass[11pt]{article}

\usepackage[brazil]{babel}
\usepackage[utf8]{inputenc}
\usepackage[usenames,dvipsnames,svgnames,table]{xcolor}
\usepackage[a4paper,margin={1in}]{geometry}
\usepackage{graphicx}
\usepackage{color}
\usepackage{courier}
\usepackage{pifont}
\usepackage{pgfplots}
\definecolor{sblue}{rgb}{0, 0, 1}
\definecolor{blue}{rgb}{0, 0.55, 1}
\definecolor{red}{rgb}{1, 0, 0}
\newcommand{\quotes}[1]{``#1''}

\renewcommand{\baselinestretch}{1.5}

\newcommand{\vsp}{\vspace{0.2in}}

\begin{document}


\centerline{
  \begin{minipage}[t]{5in}
    \begin{center}
    {\Large \bf README - EP3}
    \vsp \\
	{\small {\bf Disciplina:} Programação Concorrente - MAC0438 / IME-USP}\\
	{\small {\bf Professor:} Daniel Macêdo Batista}
    \end{center}
  \end{minipage}
}
\vsp


\section{Integrantes}

\begin{tabular}{ll}
\textbf {Nome} &  \textbf {NUSP} \\
Renan Fichberg & 7991131
\end{tabular}

\vsp

\section{Arquivos e diretórios}

O diretório \textbf{ep3-renan} deve conter os seguintes arquivos e diretórios:
\subsection{Arquivos}
\begin{itemize}
\item \textbf{Dinner.java} -- Classe que representa o jantar dos filósofos. Contém a função \textit{main}.
\item \textbf{InputReader.java} -- Classe para fazer a coleta dos dados de entradas e criar algumas entidades, como os filósofos e o monitor.
\item \textbf{Philosopher.java} -- Classe que representa um filósofo. Cada filósofo é uma \textit{thread}.
\item \textbf{Monitor.java} -- Classe que implementa um monitor para garantir exclusão mútua.
\item \textbf{State.java} -- Enum com os possíveis 2 estados dos filósofos.
\item \textbf{Chronometer.java} -- Classe para cronometrar o tempo decorrido no jantar dos filósofos.
\item \textbf{makefile} -- Para compilar o programa.
\item \textbf{Relatorio.pdf} -- Documento que contém resultados experimentais e observações.
\item \textbf{README.pdf} -- Este documento, que contém uma explicação geral do programa.
\end{itemize}

\subsection{Diretórios}
\begin{itemize}
	\item \textbf{inputs} -- Diretório que deverá conter, \textbf{obrigatoriamente}, os arquivos de entrada. Estão sendo enviados 2 arquivos de exemplos de entrada (simple.txt e complex.txt). 
\end{itemize}



\section{Compilação e execução}
Para compilar, você pode utilizar o programa \textit{make} sem nenhum parâmetro:

\begin{flushleft}
\textcolor{blue}{\$make}
\end{flushleft}

\noindent Para executar o programa gerado na compilação, \textbf{você deve passar obrigatoriamente os seguintes 3 parâmetros nessa ordem}.

\begin{flushleft}
\textcolor{blue}{\$ java Dinner $<$arquivo\_de\_entrada.txt$>$ R [U $|$ P]}

onde, 
\end{flushleft}
\begin{itemize}
\item \textbf{arquivo\_de\_entrada.txt} -- Um arquivo com \textbf{obrigatoriamente} extensão .txt no seguinte formato:
\begin{itemize}
	\item Linha 1: A quantidade $N$ de filósofos.
	\item Linha 2: Os $p_i$ pesos dos $N$ filosófos, $1 \leq i \leq N$, no formato $p_1$ $p_2$ $p_3$ ... $p_N$ 
\end{itemize} 
\item \textbf{R} -- A quantidade de porções de comida que os filósofos comerão. O jantar (programa) encerra quando este valor atingir zero.
\item \textbf{[U $|$ P]} -- Modo de operação. Há 2 possíveis modos:
\begin{enumerate}
	\item Modo \textbf{U} -- Cada filosófo $i$ comerá uma unidade de porção de comida, independente do seu peso, \textit{quando conseguir adquirir 2 garfos}.
	\item Modo \textbf{P} -- Cada filosófo $i$ comerá $p_i$ unidade(s) de porção de comida \textit{quando conseguir adquirir 2 garfos}.
\end{enumerate}
\end{itemize}


\section{Sobre o programa}

O programa foi escrito em linguagem Java pela facilidade para implementar o monitor e manipular as \textit{threads}.

\pagebreak

\subsection{Monitor}

A seguir, é apresentada a estrutura do monitor implementada no programa.

\begin{verbatim}
class Monitor{
	    Recursos que devem ser protegidos;
	    Trava;
	    Varáveis de Condição;

	    Construtor();

	    Métodos();

	    Aliases();
}
\end{verbatim}

Todos os métodos do monitor são funções públicas, e as suas funções privadas são, na verdade, apenas \textit{aliases} para seguir as especificações do enunciado. O monitor foi implementado de forma a garantir exclusão mútua para cada par de filosófos que concorrerem a um garfo (\textit{que é um recurso}). Todos os métodos do monitor, quando invocados, tentam se apoderar da trava, que só é liberada quando o filósofo (que é uma \textit{thread}) que a obteve terminou de fazer o que precisava. Ainda, internamente, os métodos usam \textit{wait} e \textit{signal} para solucionar os problemas com a trava.

Ao entrar na sala de jantar, um filósofo sabe qual é o seu lugar. A primeira coisa que ele vê é quantos garfos disponíveis ainda restam sobre a mesa. Se o número de garfos $G$ for $G \leq 1$, o filósofo aguarda até ter mais de um garfo disponível sobre a mesa. Caso $G > 1$, o filósofo olha, na seqüência se o seu garfo esquerdo está disponível. Caso não esteja, o filósofo aguarda a sua disponibilidade. Uma vez com o garfo direito obtido, o filósofo parte em busca do garfo direito. Uma vez conseguido ambos os garfos, o filósofo saboreia o seu miojo e, depois disso, avisa todos os que estavam esperando os seus garfos que agora eles estão disponíveis. Também avisa que o número de garfos na mesa já é $G > 1$.

\pagebreak
\section{Guia de implementação}

Nesta seção, iremos brevemente falar sobre as classes e os seus métodos. \textbf{\textit{Setters} e \textit{Getters} não estão escritos neste documento}.

\subsection{Classe: Dinner}
A classe que representa o jantar dos filósofos e o ponto de partida do programa. Quando a quantidade de comida \textit{food} desta classe atingir zero, o programa encerra. 
\begin{itemize}
	\item \textbf{\textcolor{sblue}{void} main} -- Função \textit{main}.
	\item \textbf{\textcolor{sblue}{boolean} have\_food} -- Verifica se a comida do jantar ainda não terminou.
\end{itemize}

\subsection{Classe: InputReader}
Classe responsável por coletar todas as informações da entrada e fazer checagens para ver se o programa está sendo alimentado corretamente. 
\begin{itemize}
	\item \textbf{\textcolor{sblue}{void} create\_philosopher} -- Função que cria um objeto filósofo e o adiciona a uma lista de filósofos.
	\item \textbf{\textcolor{sblue}{boolean} correct\_number\_of\_arguments} -- Verifica se o número de argumentos do programa é válido.
	\item \textbf{\textcolor{sblue}{boolean} good\_file} -- Verifica se o arquivo de entrada está no formato correto e coleta seus dados.
	\item \textbf{\textcolor{sblue}{boolean} good\_mode\_value} -- Verifica se o valor do modo é válido e, em caso afirmativo, o obtém.
	\item \textbf{\textcolor{sblue}{boolean} good\_food\_value} -- Verifica se o valor da comida é válido e, em caso afirmativo, o obtém.
	\item \textbf{\textcolor{sblue}{boolean} is\_integer} -- Verifica se algum valor é inteiro.
	\item \textbf{\textcolor{sblue}{boolean} good\_extension} -- Verifica se o arquivo de entrada tem a extensão correta, que é .txt.
	\item \textbf{\textcolor{sblue}{void} use} -- Imprime modo de usar.
	\item \textbf{\textcolor{sblue}{void} bad\_file} -- Imprime problemas que podem ter ocorrido com relação ao \textcolor{blue}{arquivo\_de\_entrada}.
	\item \textbf{\textcolor{sblue}{void} bad\_food\_value} -- Imprime problemas que podem ter ocorrido com relação ao valor \textcolor{blue}{R} da entrada.
	\item \textbf{\textcolor{sblue}{void} bad\_mode\_value} -- Imprime problemas que podem ter ocorrido com relação ao valor \textcolor{blue}{[U $|$ P]} da entrada.
\end{itemize}

\subsection{Classe: Philosopher}
A classe que representa os filósofos. 
\begin{itemize}
	\item \textbf{\textcolor{sblue}{void} run} -- Para rodar a \textit{thread}.
	\item \textbf{\textcolor{sblue}{void} change\_state} -- Muda o estado do filósofo.
	\item \textbf{\textcolor{sblue}{boolean} is\_thinking} -- Verifica se o filósofo está pensando.
	\item \textbf{\textcolor{sblue}{boolean} is\_eating} -- Verifica se o filósofo está comendo.
	\item \textbf{\textcolor{sblue}{void} eat} -- O filósofo se prepara para comer uma quantidade (dependente se o modo for \textcolor{blue}{[U $|$ P]}) de porção de comida do jantar.
	\item \textbf{\textcolor{sblue}{void} degustate} -- O filósofo está degustando a comida.
	\item \textbf{\textcolor{sblue}{void} digest} -- O filósofo está fazendo a digestão. Neste instante a \textit{thread} dorme por um tempo \textbf{fixo}.
	\item \textbf{\textcolor{sblue}{void} think} -- O filósofo muda seu estado para pensando e começa a se concentrar.
	\item \textbf{\textcolor{sblue}{void} focus} -- O filósofo se concentra nos seus pensamentos. Neste instante a \textit{thread} dorme por um tempo \textbf{arbitrário}.
\end{itemize}

\subsection{Enum: State}
Contém os 2 possíveis estados de um filósofo.

\subsection{Classe: Chronometer}
A classe usada para contar o tempo decorido no jantar. 
\begin{itemize}
	\item \textbf{\textcolor{sblue}{String} get\_time} -- Retorna uma string no formato AhBmCs para ser impressa. A, B e C são números inteiros e h, m e s representam hora, minuto e segundo, respectivamente.
	\item \textbf{\textcolor{sblue}{long} define\_time} -- Define os segundos decorridos de determinado minuto.
\end{itemize}

\subsection{Classe: Monitor}
A classe que implementa o monitor. 
\begin{itemize}
	\item \textbf{\textcolor{sblue}{void} get\_forks} -- O filósofo tenta pegar o seu par de garfos.
	\item \textbf{\textcolor{sblue}{void} put\_forks} -- O filósofo devolve o par de garfos que pegou.
	\item \textbf{\textcolor{sblue}{void} wait} -- \textit{Alias}.
	\item \textbf{\textcolor{sblue}{void} signal} -- \textit{Alias}.
\end{itemize}


\end{document}